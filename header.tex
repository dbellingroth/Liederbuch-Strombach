\usepackage{ngerman}
\usepackage[T1]{fontenc}
\usepackage{xltxtra}
\usepackage{tikz}
\usepackage{xstring}
\usepackage[inner=14mm, outer=7mm, top=7mm, bottom=7mm]{geometry}
\usepackage{tocloft}
\usepackage{makeidx}
\usepackage[columns=2,indentunit=0mm,justific=raggedright]{idxlayout}
\usepackage{letltxmacro}
\usepackage{pdfpages}
\usepackage[colorlinks=false, bookmarks, hyperindex=false]{hyperref}
\usepackage{keyval}

%Style für xindy
\begin{filecontents}[overwrite,noheader]{inhalt.xdy}
(markup-keyword-list :open "\hangindent=2.5mm \hangafter=1" :close "" :depth 0)
(markup-locclass-list :open "\nobreak\dotfill{}" :close "")
(markup-letter-group :open-head "~n~n \textbf{\textsf{" :close-head "}}\nopagebreak~n")
\end{filecontents} 

\renewcommand{\indexname}{Inhaltsverzeichnis}

\newcounter{songnumber}
\setcounter{songnumber}{1}

\newcounter{songpage}
\setcounter{songpage}{1}

\newcounter{realpage}

\makeatletter
\define@key{song}{transform}{\def\transform{#1}}
\define@key{song}{pagecommand}{\def\pagecommand{#1}}
\setkeys{song}{transform=plain,pagecommand={}}
\makeatother

% Kommando um Lieder einzufügen
\newcommand{\song}[3][]{{

\setkeys{song}{#1}


% Skalierung und Verschiebung der PDF-Datei festlegen
  \ifthenelse{\equal{#1}{old}}{  
    \newcommand{\yoffset}{-8}
    \newcommand{\scale}{1.1}
  }{
    \ifthenelse{\equal{#1}{new}}{  
      \newcommand{\yoffset}{-4mm}
      \newcommand{\scale}{1.06}
    }{
      \newcommand{\yoffset}{0}
      \newcommand{\scale}{1}
    }
  }

% Ersten Titel auslesen
\StrCount{#2}{,}[\numoftitles]

\ifnum\numoftitles>0\relax
    \StrBefore[1]{#2}{,}[\songtitle]%
\else
    \def\songtitle{#2}%
\fi


% Echte Seitennummer merken
\setcounter{realpage}{\arabic{page}}
% Seitencounter auf Liednummer setzen, um Indexerstellung auszutricksen   
\setcounter{page}{\arabic{songnumber}}
% Titel zu Index hinzufügen
\foreach \titel in {#2} {
	\index{\titel|hyperpage}
}

% Lied-PDF einbinden
\includepdf[pages=-,offset= 0 \voffset, scale=\scale,pagecommand={
% Alle Seiten eines Liedes bekommen die gleiche Seitennummer; die Liednummer. Damit das Fortlaufen innerhalb eines Liedes unterbunden wird und man beim Index (Inhaltsverzeichnis) zum richtigen Lied springt (erste Seite mit der entsprechenden Seitennummer).
\setcounter{page}{\arabic{songnumber}}
\ifnumcomp{\value{songpage}}{<}{2}
{ % Erste Seite des Lieds
	\begin{tikzpicture}[remember picture, overlay]
		% Bookmark im PDF erstellen
		\node[below=0mm] at (current page.north){
			\hypertarget{\arabic{songnumber}}{}
			\pdfbookmark[1]{Lied \arabic{songnumber} - \songtitle}{\arabic{songnumber}}
		};
        % Liednummer einfügen
        \ifnumodd{\value{realpage}}{
        	\node[below=11.1mm, left=7mm][anchor=north east] at (current page.north east){
        		\textsf{\Huge{\hyperlink{Inhaltsverzeichnis}{\textbf{\arabic{songnumber}}}}}
        	};
        }{
        	\node[below=11.1mm, right=7mm][anchor=north west] at (current page.north west){
        		\textsf{\Huge{\hyperlink{Inhaltsverzeichnis}{\textbf{\arabic{songnumber}}}}}
        	};
        }
     \end{tikzpicture}
}{ 
  % Jede weitere Seite des Lieds
}
\pagecommand
\stepcounter{songpage}
\stepcounter{realpage}
}]{noten/#3}
\setcounter{songpage}{1}
\stepcounter{songnumber}

% Auf echte Seitenzahl zurücksetzen
\setcounter{page}{\arabic{realpage}}
\newpage
}}

\makeatletter
\define@key{songimage}{x}{\def\x{#1}}
\define@key{songimage}{y}{\def\y{#1}}
\define@key{songimage}{scale}{\def\scale{#1}}
\define@key{songimage}{page}{\def\page{#1}}
\makeatother

\newcommand{\songimage}[2][]{
\setkeys{songimage}{page=1,x=0,y=0,scale=1}
\setkeys{songimage}{#1}
\ifnum\thesongpage=\page
\begin{tikzpicture}[remember picture, overlay]
		\node[right=\x,below=\y] at (current page.north){
			\includegraphics[scale=\scale]{#2}
		};
     \end{tikzpicture}

\fi
}

\newcommand{\emptypage}[1]{
	\newpage
	\thispagestyle{plain} % empty
	\mbox{#1}
}

\newcommand{\textpage}[1]{
	#1
  \newpage
}

\makeindex
